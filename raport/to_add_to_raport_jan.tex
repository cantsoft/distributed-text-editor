\Week{Dataset Preparation for Synchronization Testing}

\subsection{Purpose}

To evaluate the behavior of our distributed text editor under realistic editing conditions, we required
a dataset that reflects natural language usage instead of artificially generated sequences. The dataset
is intended to support testing of synchronization logic, conflict handling and convergence properties.

\subsection{Data Source Selection}

We chose Wikipedia as the data source due to:
\begin{itemize}
    \item public accessibility and a stable API for automated retrieval,
    \item sufficiently complex and natural sentence structure,
    \item extensive internal linking, making it easy to gather thematically related material.
\end{itemize}

An initial article is located through a keyword search, and additional articles are collected by following
internal links, resulting in a coherent yet diverse text sample.

\subsection{Text Representation}

The combined article text is transformed into a sequence of elementary editing operations. Each character
is represented as an independent entry, associated with:
\begin{itemize}
    \item its position in the text,
    \item the type of change (insertion or deletion),
    \item a timestamp representing logical operation ordering.
\end{itemize}

Since editing in our system is performed by application instances rather than multiple human users, we do
not differentiate between user identities at this stage.

\subsection{Editing Scenarios}

To test synchronization behavior in different conditions, we prepared several controlled editing scenarios:
\begin{itemize}
    \item \textbf{Gradual text construction:} characters are added sequentially, simulating normal typing.
    \item \textbf{Progressive text removal:} characters are removed step-by-step, simulating revision or rollback.
    \item \textbf{Concurrent modification:} multiple operations share the same timestamp, simulating simultaneous edits.
\end{itemize}

These scenarios help evaluate whether the editor maintains a consistent document state in the presence of
concurrent operations and conflicting modifications.

\subsection{Outcome}

The result is a reproducible dataset based on real text, structured as a sequence of timestamped editing
operations. This dataset will be used in further development to:
\begin{itemize}
    \item assess the correctness of the synchronization algorithm,
    \item evaluate convergence under concurrent editing conditions,
    \item observe system behavior during large-scale modifications.
\end{itemize}
